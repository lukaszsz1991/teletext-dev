\documentclass[12pt]{article}
\usepackage[T1]{fontenc}
\usepackage[polish]{babel}
\usepackage[utf8]{inputenc}
\usepackage{graphicx}
\usepackage{float}
\usepackage[colorlinks=true,
linkcolor=black,
urlcolor=blue,
citecolor=black]{hyperref}



%opening
\title{Dokumentacja użytkownika aplikacji Teletext}
\author{Sebastian Górski\\ Jakub Grzymisławski\\ Łukasz Szenkiel\\Rafał Wilczewski}
\date{}


\begin{document}
	
	\maketitle
	\thispagestyle{empty} 
	
	\begin{center}
		\vspace{10em}
		\includegraphics[width=0.6\textwidth]{CWUPlogo.png} \\[10em]
	\end{center}
	
	\newpage
	\tableofcontents
	
	\newpage
\section*{Opis}
\addcontentsline{toc}{section}{Opis}

Dokumentacja dotyczy aplikacji \textit{Teletext}, nowoczesnego systemu zarządzania telegazetą. Aplikacja została zaprojektowana z myślą o ułatwieniu tworzenia, edycji i publikacji treści telegazety w~sposób szybki, intuicyjny i zgodny ze współczesnymi standardami cyfrowymi.  

System umożliwia użytkownikom zarówno przeglądanie dostępnych stron telegazety, jak i zarządzanie jej zawartością w czasie rzeczywistym. Względem aplikacji postawiono następujące wymagania:  
\begin{itemize}
	\item użytkownik może przeglądać strony telegazety (struktura numerów stron!)
	\item użytkownik może wyszukiwać informacje po tytułach i kategoriach
	administrator może tworzyć własne strony telegazety
	\item strona ma numer, tytuł, kategorię i treść
	\item treść może zawierać tekst i proste elementy graficzne (ASCII)
	\item administrator może przypisywać wybrane integracje do numerów stron
	\item należy zaimplementować min. 7 integracji, np. pogoda, wyniki lotto, głosowania w Sejmie, ogłoszenia o pracę, kursy walut, ceny kruszców
	\item administrator widzi statystyki najczęściej odwiedzanych stron
\end{itemize}

Celem projektu jest stworzenie narzędzia, które nie tylko ułatwi pracę redaktorom i administratorom telegazety, ale również zapewni użytkownikom końcowym przyjazny i funkcjonalny interfejs do przeglądania informacji.

\newpage
\section{Wymagania systemowe}
\subsection{Wymagania sprzętowe}

Aplikacja \textit{Teletext} została zaprojektowana jako system działający w przeglądarce internetowej, dlatego nie wymaga specjalistycznego sprzętu ani instalacji dodatkowego oprogramowania. Do prawidłowego działania aplikacji wystarczy dowolne urządzenie (komputer stacjonarny, laptop, tablet lub smartfon) wyposażone w nowoczesną przeglądarkę internetową oraz stabilne połączenie z Internetem.  

Dzięki temu użytkownicy mogą korzystać z aplikacji zarówno w środowisku domowym, jak i w pracy, bez konieczności spełniania konkretnych parametrów sprzętowych.

\subsection{Wymagania programowe}

\begin{itemize}
	\item \textbf{System operacyjny:} dowolny system obsługujący nowoczesną przeglądarkę internetową (Windows, macOS, Linux, Android, iOS)
	\item \textbf{Przeglądarka internetowa:} Google Chrome, Mozilla Firefox, Microsoft Edge, Safari (wersje aktualne)
	\item \textbf{Aktywne JavaScript i obsługa cookies} w przeglądarce
	\item \textbf{Połączenie internetowe:} wymagane do dostępu do serwera aplikacji
\end{itemize}


\section{Role użytkowników}
Aplikacja \textit{Teletext} przewiduje różne poziomy dostępu w zależności od uprawnień użytkownika. Każda rola ma przypisane konkretne funkcje i możliwości, co pozwala zapewnić bezpieczeństwo systemu oraz odpowiednią kontrolę nad treścią telegazety. Wyróżniamy następujące role:

\subsection{Użytkownik końcowy}

	\textbf{Użytkownik końcowy} – osoba przeglądająca strony telegazety. Użytkownik może:
	\begin{itemize}
		\item przeglądać dostępne strony według numerów i kategorii,
		\item wyszukiwać treści po tytułach i kategoriach,
		\item korzystać z integracji (np. pogoda, kursy walut) udostępnionych na stronach.
	\end{itemize}
	
	\subsection{Administrator}
	\textbf{Administrator / Redaktor} – osoba odpowiedzialna za tworzenie, edycję i~publikację treści. Administrator może:
	\begin{itemize}
		\item tworzyć nowe strony telegazety,
		\item edytować lub usuwać istniejące strony,
		\item przypisywać integracje do wybranych numerów stron,
		\item przeglądać statystyki odwiedzin stron,
		\item zarządzać kategoriami i tytułami stron.
	\end{itemize}


\subsection{Diagram przypadków użycia}

Poniżej przedstawiono diagram przypadków użycia, ilustrujący interakcje pomiędzy użytkownikami a systemem:

	\begin{figure}[H]
	\centering
	\includegraphics[width=0.9\linewidth]{docimages/diagram_przypadkow_uzycia.png}
	\caption{Diagram przypadków użycia}
	\label{fig:diagram_przypadkow_uzycia}
	\end{figure}

\noindent
Na diagramie widać dwie główne role: \textit{Użytkownik końcowy} oraz \textit{Administrator}, oraz ich interakcje z funkcjami systemu, takimi jak przeglądanie stron, wyszukiwanie informacji, zarządzanie treścią i monitorowanie statystyk.


\section{Uruchomienie aplikacji}

Aplikacja \textit{Teletext} została stworzona jako system działający w przeglądarce internetowej, dlatego jej uruchomienie nie wymaga instalacji dodatkowego oprogramowania na komputerze użytkownika. Poniżej przedstawiono kroki niezbędne do rozpoczęcia korzystania z aplikacji.

\subsection{Dostęp do aplikacji}

\begin{enumerate}
	\item Otwórz przeglądarkę internetową (Google Chrome, Mozilla Firefox, Microsoft Edge, Safari).  
	\item W pasku adresu wpisz adres aplikacji:  
\begin{center}
	\url{http://34.118.43.40/}
\end{center}
	\item Po załadowaniu strony użytkownik zostaje przekierowany do ekranu głównego, z którego można przeglądać dostępne strony telegazety.
\end{enumerate}

\subsection{Logowanie administratora}

Aby uzyskać dostęp do funkcji administracyjnych (tworzenie i edycja stron, przypisywanie integracji, podgląd statystyk), należy zalogować się jako administrator:

\begin{enumerate}
	\item Na ekranie głównym kliknij przycisk „Zaloguj się” lub przejdź do sekcji logowania.  
	\item Wprowadź swoją nazwę użytkownika i hasło.  
	\item Kliknij przycisk „Zaloguj się”.  
	\item Po poprawnym zalogowaniu system przekieruje użytkownika do panelu administratora, gdzie dostępne są wszystkie funkcje zarządzania treścią.
\end{enumerate}

\subsection{Wylogowanie}

Po zakończeniu pracy w panelu administratora należy się wylogować, aby chronić dane i ograniczyć dostęp do funkcji zarządzania:

\begin{enumerate}
	\item Kliknij przycisk „Wyloguj” w górnym menu panelu administratora.  
	\item System przekieruje użytkownika z powrotem na ekran główny aplikacji.
\end{enumerate}


\subsection{Wymagania dodatkowe dla uruchomienia}
	\begin{itemize}
	\item Aktywne połączenie internetowe.  
	\item Włączona obsługa JavaScript w przeglądarce.  
	\item Włączona obsługa cookies, jeśli logowanie administratora opiera się na sesjach.
	\end{itemize}

\noindent
Dzięki powyższym krokom użytkownicy końcowi mogą natychmiast rozpocząć przeglądanie telegazety, a administratorzy zarządzanie jej zawartością w pełni funkcjonalny sposób.


\section{Interfejs użytkownika}
\subsection{Ekran główny}

Ekran główny aplikacji \textit{Teletext} jest centralnym miejscem startowym zarówno dla użytkowników końcowych, jak i administratorów. Został zaprojektowany w stylu retro, charakterystycznym dla klasycznych telegazet, i zapewnia przejrzystą nawigację oraz szybki dostęp do najważniejszych funkcji systemu.

	\begin{figure}[H]
	\centering
	\includegraphics[width=0.9\linewidth]{docimages/ekran_glowny.png}
	\caption{Ekran główny}
	\label{fig:ekran_glownby}
	\end{figure}

\begin{itemize}
	\item \textbf{Nagłówek:} 
	\begin{itemize}
		\item Wyświetla nazwę aplikacji: \textbf{TELEGAZETA}  
		\item Podtytuł informujący o systemie zarządzania: \textbf{SYSTEM ZARZĄDZANIA TELEGAZETĄ}
	\end{itemize}
	
	\item \textbf{Portal Czytelnika:}
	\begin{itemize}
		\item Sekcja dedykowana użytkownikom końcowym, oznaczona ikoną \texttt{Portal Czytelnika}.
		\item Zawiera przycisk \textbf{CZYTAJ}, umożliwiający przeglądanie numerów telegazety oraz odkrywanie nowych artykułów.
		\item Krótki opis funkcjonalności: „Przeglądaj numery telegazety, odkrywaj nowe artykuły i bądź na bieżąco z najnowszymi wiadomościami.”
	\end{itemize}
	
	\item \textbf{Panel Administratora:}
	\begin{itemize}
		\item Sekcja dla administratorów, oznaczona ikoną \texttt{Panel Administratora}.
		\item Zawiera przycisk \textbf{ZALOGUJ}, który umożliwia dostęp do funkcji zarządzania treścią.
		\item Administrator może tworzyć i edytować strony telegazety, przypisywać integracje oraz monitorować statystyki odwiedzin.
		\item Sekcja jest wyraźnie oddzielona od portalu czytelnika, co zapobiega przypadkowemu wejściu w tryb administracyjny.
	\end{itemize}
	
	\item \textbf{Co oferujemy:} 
	\begin{itemize}
		\item Przegląd struktury numerów telegazety.
		\item Edytor stron z możliwością dodawania tekstu i elementów ASCII art.
		\item Integracje zewnętrzne: pogoda, wyniki lotto, tabela ekstraklasy, kursy walut, wiadomości.
		\item Statystyki odwiedzin i popularności stron.
		\item Responsywny design w stylu retro.
	\end{itemize}
	
	\item \textbf{Technologia i informacje o systemie:}
	\begin{itemize}
		\item Frontend: React
		\item Komunikacja z backendem: REST API
		\item Dokumentacja API: OpenAPI
		\item Styl graficzny: klasyczna telegazeta (retro aesthetic)
	\end{itemize}
	
	\item \textbf{Stopka:} 
	\begin{itemize}
		\item Informacje o prawach autorskich: \texttt{TELEGAZETA © 2025 | Wszystkie prawa zastrzeżone}  
		\item Wersja systemu: \texttt{Wersja 1.0 BETA | System v0.1}
	\end{itemize}
\end{itemize}

Ekran główny zapewnia intuicyjną i szybką nawigację zarówno dla użytkowników końcowych, jak i administratorów, jednocześnie zachowując charakterystyczny styl retro, który odzwierciedla klasyczną telegazetę.

\subsection{Nawigacja po stronach telegazety}

Sekcja \textbf{Lista Stron Telegazety} umożliwia użytkownikom końcowym szybkie przeglądanie dostępnych stron oraz dostęp do aktualnych informacji w~różnych kategoriach. Interfejs został zaprojektowany w sposób przejrzysty i~intuicyjny, zachowując charakterystyczny styl retro.

	\begin{figure}[H]
	\centering
	\includegraphics[width=0.9\linewidth]{docimages/przegladanie_stron.png}
	\caption{Przeglądanie stron}
	\label{fig:przegladanie_stron}
	\end{figure}

\begin{itemize}
	\item \textbf{Nagłówek:} 
	\begin{itemize}
		\item Wyświetla nazwę aplikacji: \textbf{TELEGAZETA}  
		\item Podtytuł informujący o systemie zarządzania: \textbf{SYSTEM ZARZĄDZANIA TELEGAZETĄ}
	\end{itemize}
	
	\item \textbf{Lista stron:} 
	\begin{itemize}
		\item Każda strona wyświetlana jest z numerem, tytułem oraz kategorią, np.:  
		\begin{itemize}
			\item 101 – Wiadomość lokalna (Wiadomości)  
			\item 102 – Najnowsze Wiadomości (LIVE) (Wiadomości)  
			\item 201 – Tabela Ekstraklasa (Sport)  
			\item 302 – Wyniki Lotto (LIVE) (Gry losowe)  
			\item 501 – Pogoda Wrocław (Pogoda)  
			\item 801 – Kursy Walut (LIVE) (Finanse)  
			\item 901 – Dodatkowa informacja wprowadzona przez administratora (Różne)
		\end{itemize}
		\item Łączna liczba stron jest wyświetlana na dole listy, np. „Łącznie stron: 10”.
		\item Strony oznaczone jako \texttt{LIVE} są aktualizowane w czasie rzeczywistym.
	\end{itemize}
	
	\item \textbf{Wyświetlanie strony:} 
	\begin{itemize}
		\item Kliknięcie numeru strony powoduje otwarcie pełnej zawartości danej strony telegazety.
		\item Strona może zawierać tekst, elementy ASCII art oraz integracje (pogoda, kursy walut, wyniki lotto itp.).
	\end{itemize}
	
	\item \textbf{Nawigacja:} 
	\begin{itemize}
		\item Użytkownik może wrócić do ekranu głównego klikając przycisk „\texttt{← Strona główna}”.
		\item Interfejs jest responsywny, co pozwala na wygodne przeglądanie stron zarówno na komputerach, jak i urządzeniach mobilnych.
	\end{itemize}
	
	\item \textbf{Stopka:} 
	\begin{itemize}
		\item Zawiera informacje o prawach autorskich: \texttt{TELEGAZETA © 2025 | Wszystkie prawa zastrzeżone}  
		\item Wersja systemu: \texttt{Wersja 1.0 BETA | System v0.1}
	\end{itemize}
\end{itemize}

Ekran \textbf{Lista Stron Telegazety} zapewnia użytkownikom szybki dostęp do wszystkich numerów telegazety, umożliwia natychmiastowe wyświetlenie interesujących treści i korzystanie z aktualnych integracji w czasie rzeczywistym.

\subsection{Panel administratora}

Panel administratora jest przeznaczony dla użytkowników z uprawnieniami administracyjnymi i umożliwia pełne zarządzanie treścią telegazety oraz monitorowanie jej statystyk. Dostęp do panelu wymaga uprzedniego zalogowania.

\subsubsection{Logowanie administratora}

\begin{enumerate}
	\item Na ekranie głównym aplikacji kliknij przycisk \textbf{ZALOGUJ} w sekcji \texttt{Panel Administratora}.  
	\item W formularzu logowania wprowadź swoją nazwę użytkownika oraz hasło.  
	\item Kliknij przycisk \textbf{Zaloguj się}.  
	\item Po poprawnym uwierzytelnieniu system przekieruje administratora do panelu, gdzie dostępne są wszystkie funkcje zarządzania.  
\end{enumerate}

	\begin{figure}[H]
	\centering
	\includegraphics[width=0.9\linewidth]{docimages/okno_logowania.png}
	\caption{Okno logowania}
	\label{fig:okno_logowania}
	\end{figure}

\noindent
\textbf{Uwagi:}
\begin{itemize}
	\item Nieudane próby logowania skutkują wyświetleniem komunikatu o~błędzie.  
	\item Logowanie chroni dostęp do edycji treści i konfiguracji integracji.  
	\item Po zakończeniu pracy zaleca się wylogowanie, aby zabezpieczyć panel przed nieautoryzowanym dostępem.
\end{itemize}

\subsubsection{Funkcje panelu administratora}

Panel administratora oferuje szereg narzędzi do zarządzania zawartością telegazety:

\begin{itemize}
	\item \textbf{Tworzenie i edycja stron:}  
	Administrator może tworzyć nowe strony telegazety, przypisywać im numer, tytuł, kategorię oraz wprowadzać treść. Edytor wspiera również dodawanie prostych elementów graficznych w formacie ASCII art.
	
	\item \textbf{Zarządzanie integracjami:}  
	Panel umożliwia przypisywanie wybranych integracji do konkretnych stron, np.: pogoda, wyniki lotto, tabela ekstraklasy, kursy walut, wiadomości. Administrator może aktywować, wyłączyć lub skonfigurować parametry integracji.
	
	\item \textbf{Monitorowanie statystyk:}  
	Panel prezentuje statystyki odwiedzin stron, w tym listę najczęściej przeglądanych treści. Pozwala to administratorom analizować popularność stron i optymalizować układ oraz zawartość telegazety.
	
	\item \textbf{Nawigacja w panelu:}  
	Panel jest podzielony na sekcje umożliwiające szybkie przełączanie się między tworzeniem/edycją stron, konfiguracją integracji oraz podglądem statystyk.  
	Przycisk „Wyloguj” umożliwia bezpieczne zakończenie sesji.
\end{itemize}

\subsubsection{Wygląd panelu}

- Panel administratora jest zaprojektowany w stylu zgodnym z retro estetyką całej aplikacji, podobnie jak ekran główny.  
- Interfejs jest responsywny, dzięki czemu umożliwia zarządzanie treścią zarówno na komputerach stacjonarnych, jak i na urządzeniach mobilnych.  
- Kluczowe przyciski i sekcje są wyraźnie oznaczone, co ułatwia szybkie wykonywanie zadań administracyjnych.

	\begin{figure}[H]
		\centering
		\includegraphics[width=1.1\linewidth]{docimages/panel_administratora.png}
		\caption{Wygląd panelu administratora}
		\label{fig:panel_administratora}
	\end{figure}

\noindent
Dzięki tym funkcjom panel administratora stanowi centralne narzędzie do tworzenia i utrzymywania telegazety, zapewniając jednocześnie bezpieczeństwo i pełną kontrolę nad zawartością systemu.


\section{Instrukcja użytkowania}

Niniejszy rozdział opisuje sposób korzystania z aplikacji \textit{Teletext} zarówno przez użytkowników końcowych, jak i administratorów systemu. Instrukcja została podzielona na podsekcje odpowiadające poszczególnym rolom użytkowników.

\subsection{Instrukcja dla użytkownika końcowego}

Użytkownik końcowy ma dostęp do funkcji przeglądania stron telegazety oraz korzystania z dostępnych integracji informacyjnych.

\subsubsection{Uruchomienie aplikacji}

Aby rozpocząć korzystanie z aplikacji:
\begin{enumerate}
	\item Otwórz dowolną nowoczesną przeglądarkę internetową.
	\item Wpisz adres aplikacji \url{http://34.118.43.40/} w pasku adresu przeglądarki.
	\item Po załadowaniu strony wyświetlony zostanie ekran główny aplikacji.
\end{enumerate}

\subsubsection{Przeglądanie stron telegazety}

\begin{enumerate}
	\item Na ekranie głównym kliknij przycisk \textbf{CZYTAJ} w sekcji \texttt{Portal Czytelnika}.
	\item Zostanie wyświetlona lista dostępnych stron telegazety.
	\item Każda strona opisana jest numerem, tytułem oraz kategorią.
	\item Kliknięcie numeru strony powoduje otwarcie jej pełnej zawartości.
\end{enumerate}

\subsubsection{Korzystanie z integracji}

Niektóre strony telegazety zawierają integracje z zewnętrznymi źródłami danych, oznaczone jako \texttt{LIVE}.
Użytkownik może:
\begin{itemize}
	\item sprawdzać aktualną pogodę,
	\item przeglądać kursy walut,
	\item odczytywać wyniki losowań i wydarzeń sportowych,
	\item korzystać z dynamicznie aktualizowanych treści.
\end{itemize}

Dane na stronach oznaczonych jako \texttt{LIVE} są aktualizowane automatycznie.

\subsubsection{Nawigacja}

\begin{itemize}
	\item Powrót do listy stron możliwy jest poprzez przycisk \texttt{← Strona główna}.
	\item Aplikacja działa w sposób responsywny i dostosowuje się do rozmiaru ekranu urządzenia.
\end{itemize}

\subsection{Instrukcja dla administratora}

Administrator posiada rozszerzone uprawnienia umożliwiające zarządzanie zawartością telegazety.

\subsubsection{Logowanie}

\begin{enumerate}
	\item Na ekranie głównym kliknij przycisk \textbf{ZALOGUJ} w sekcji \texttt{Panel Administratora}.
	\item Wprowadź nazwę użytkownika oraz hasło.
	\item Po poprawnym uwierzytelnieniu nastąpi przekierowanie do panelu administratora.
\end{enumerate}

\subsubsection{Tworzenie nowej strony telegazety}

	\begin{figure}[H]
	\centering
	\includegraphics[width=1.1\linewidth]{docimages/dodawanie.png}
	\caption{Dodawanie stron}
	\label{fig:dodawanie_stron}
	\end{figure}

Aby dodać nową stronę:
\begin{enumerate}
	\item Przejdź do sekcji zarządzania stronami.
	\item Wybierz opcję \textbf{Dodaj nową stronę}.
	\item Uzupełnij wymagane pola: numer strony, tytuł, kategoria oraz treść.
	\item Zapisz stronę, aby została opublikowana w systemie.
\end{enumerate}

\subsubsection{Edycja i usuwanie stron}

	\begin{figure}[H]
	\centering
	\includegraphics[width=1.1\linewidth]{docimages/strony.png}
	\caption{Widok stron}
	\label{fig:widok_stron}
	\end{figure}

Administrator może:
\begin{itemize}
	\item edytować istniejące strony (zmiana treści, kategorii, numeru),
	\item usuwać strony, które nie są już aktualne,
	\item aktualizować zawartość stron dynamicznych.
\end{itemize}

\subsubsection{Zarządzanie integracjami}

	\begin{figure}[H]
	\centering
	\includegraphics[width=1.1\linewidth]{docimages/integracje.png}
	\caption{Zarządzanie integracjami}
	\label{fig:integracje}
	\end{figure}

\begin{enumerate}
	\item W panelu administratora wybierz sekcję integracji.
	\item Przypisz wybraną integrację do konkretnej strony.
	\item Zapisz zmiany, aby integracja była widoczna dla użytkowników końcowych.
\end{enumerate}

\subsubsection{Podgląd statystyk}

	\begin{figure}[H]
	\centering
	\includegraphics[width=1.1\linewidth]{docimages/statystyki.png}
	\caption{Podgląd statystyk}
	\label{fig:statystyki}
	\end{figure}

Administrator ma dostęp do statystyk odwiedzin, które umożliwiają:
\begin{itemize}
	\item analizę popularności poszczególnych stron,
	\item identyfikację najczęściej odwiedzanych treści,
	\item optymalizację struktury telegazety.
\end{itemize}

\subsubsection{Wylogowanie}

Po zakończeniu pracy administrator powinien:
\begin{enumerate}
	\item Kliknąć przycisk \textbf{Wyloguj}.
	\item Zakończyć sesję w celu zabezpieczenia dostępu do panelu.
\end{enumerate}


\section{Obsługa błędów}

Aplikacja \textit{Teletext} została zaprojektowana tak, aby użytkownik w przypadku błędów otrzymywał jasne komunikaty i mógł bezpiecznie kontynuować pracę.

\subsection{Wyświetlenie nieistniejącej strony}

	\begin{figure}[H]
	\centering
	\includegraphics[width=1.1\linewidth]{docimages/brak_strony.png}
	\caption{Próba wyświetlenia strony, która nie istnieje}
	\label{fig:brak_strony}
	\end{figure}

Jeżeli użytkownik próbuje otworzyć stronę telegazety o numerze, który nie istnieje:
\begin{itemize}
	\item system wyświetla komunikat informujący, że dana strona nie istnieje,
	\item użytkownik może wrócić do listy stron telegazety i wybrać inną stronę.
\end{itemize}

\subsection{Błędy logowania administratora}

	\begin{figure}[H]
	\centering
	\includegraphics[width=0.98\linewidth]{docimages/blad_logowania.png}
	\caption{Błąd logowania: brak loginu i/lub hasła}
	\label{fig:blad_logowania1}
	\end{figure}

	\begin{figure}[H]
	\centering
	\includegraphics[width=0.98\linewidth]{docimages/blad_logowania2.png}
	\caption{Błąd logowania: hasło zbyt krótkie}
	\label{fig:blad_logowania2}
	\end{figure}

	\begin{figure}[H]
	\centering
	\includegraphics[width=1.1\linewidth]{docimages/blad_logowania3.png}
	\caption{Błąd logowania: błędne hasło}
	\label{fig:blad_logowania3}
	\end{figure}

Podczas logowania do panelu administratora mogą wystąpić różne rodzaje błędów:

\begin{itemize}
	\item \textbf{Brak loginu lub hasła:}  
	Jeśli użytkownik nie wprowadzi loginu lub hasła, system wyświetla komunikat: \textit{„Proszę wprowadzić login i hasło”}.
	
	\item \textbf{Nieistniejący login:}  
	W przypadku, gdy wprowadzony login nie istnieje w bazie danych, system wyświetla komunikat: \textit{„Nieprawidłowy login”}.
	
	\item \textbf{Błędne hasło:}  
	Jeżeli login istnieje, ale hasło jest niepoprawne, system wyświetla komunikat: \textit{„Nieprawidłowe hasło”}.
	
	\item \textbf{Hasło za krótkie:}  
	Podczas zmiany lub tworzenia hasła system sprawdza jego długość. Jeśli hasło jest zbyt krótkie, wyświetlany jest komunikat: \textit{„Hasło jest zbyt krótkie – minimum 8 znaków”}.
\end{itemize}

\subsection{Niepoprawne wypełnienie formularza dodawania strony}

	\begin{figure}[H]
	\centering
	\includegraphics[width=1.1\linewidth]{docimages/blad_dodawania_strony.png}
	\caption{Błąd dodawania strony}
	\label{fig:blad_dodawania}
	\end{figure}


Podczas tworzenia nowej strony telegazety administrator musi wypełnić wszystkie wymagane pola (numer strony, tytuł, kategoria). Jeśli pola są niekompletne lub błędnie wypełnione:
\begin{itemize}
	\item system wyświetla komunikat informujący o brakujących lub niepoprawnych danych,
	\item zapis strony zostaje zablokowany do momentu poprawnego wypełnienia formularza.
\end{itemize}


\section{Podsumowanie}
Dokumentacja przedstawia aplikację \textit{Teletext}, nowoczesny system do zarządzania telegazetą, który został zaprojektowany z myślą o prostocie użytkowania, elastyczności oraz zgodności ze współczesnymi standardami cyfrowymi.  

Projekt realizuje główne cele, takie jak:  
\begin{itemize}
	\item umożliwienie użytkownikom końcowym przeglądania i wyszukiwania informacji w telegazecie w sposób szybki i intuicyjny,  
	\item zapewnienie administratorom i redaktorom wygodnego panelu do tworzenia, edycji i publikacji stron telegazety,  
	\item integracja z zewnętrznymi źródłami danych (pogoda, kursy walut, wyniki lotto, ogłoszenia itp.),  
	\item monitorowanie statystyk odwiedzin, co pozwala na analizę popularności treści.  
\end{itemize}

Aplikacja \textit{Teletext} działa w przeglądarce internetowej, dzięki czemu jest dostępna na dowolnym urządzeniu – komputerze, laptopie, tablecie czy smartfonie – bez konieczności instalacji dodatkowego oprogramowania.  

System został zaprojektowany w sposób skalowalny i elastyczny, co umożliwia dalszy rozwój oraz dodawanie nowych funkcji i integracji w przyszłości. Dzięki temu projekt stanowi nowoczesne, funkcjonalne narzędzie zarówno dla redaktorów, administratorów, jak i zwykłych użytkowników, zapewniając wygodny dostęp do informacji i pełną kontrolę nad zawartością telegazety.


\section{Autorzy}
Zespół twórców tworzą:
\begin{itemize}
	\item Sebastian Górski -- \url{https://github.com/sgroski00}
	\item Jakub Grzymisławski -- \url{https://github.com/jgrzymislawski}
	\item Łukasz Szenkiel -- \url{https://github.com/lukaszsz1991}
	\item Rafał Wilczewski -- \url{https://github.com/Rafal-wq}
\end{itemize}

\end{document}